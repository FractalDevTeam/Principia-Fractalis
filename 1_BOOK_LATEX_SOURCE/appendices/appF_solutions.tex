\chapter{Selected Exercise Solutions}
\label{app:solutions}

This appendix provides detailed solutions to selected exercises from each chapter. Full solutions are available online.

\section{Chapter 1: Numbers and Patterns}

\textbf{Exercise 1.3:} Prove that $S_3(3n) = S_3(n)$ for all $n \in \mathbb{N}$.

\textbf{Solution:}

Let $n$ have base-3 representation: $n = \sum_{k=0}^m a_k 3^k$ where $a_k \in \{0,1,2\}$.

Then: $3n = \sum_{k=0}^m a_k 3^{k+1} = \sum_{k=1}^{m+1} a_{k-1} 3^k$

This shifts all digits left by one position, adding a zero in the ones place.

Therefore:
\begin{align}
S_3(3n) &= \sum_{k=1}^{m+1} a_{k-1} \\
&= \sum_{k=0}^m a_k \\
&= S_3(n)
\end{align}

This proves the scaling property.

\section{Chapter 17: Riemann Hypothesis}

\textbf{Exercise 16.5:} Show that if $\zeta(s) = 0$ for $\Re(s) > 1/2$, then $\zeta(1-\bar{s}) = 0$ (functional equation consequence).

\textbf{Solution:}

Functional equation: $\zeta(s) = 2^s \pi^{s-1} \sin(\pi s/2) \Gamma(1-s) \zeta(1-s)$

If $\zeta(s_0) = 0$ for $\Re(s_0) > 1/2$:

\begin{align}
0 &= 2^{s_0} \pi^{s_0-1} \sin(\pi s_0/2) \Gamma(1-s_0) \zeta(1-s_0)
\end{align}

Since $\sin(\pi s_0/2) \neq 0$ (not at integer) and $\Gamma(1-s_0) \neq 0$, we must have:
$$\zeta(1-s_0) = 0$$

Taking conjugate: $\zeta(1-\bar{s_0}) = 0$

Since $\Re(s_0) > 1/2$, we have $\Re(1-\bar{s_0}) = 1 - \Re(s_0) < 1/2$.

This violates the Riemann Hypothesis (all zeros on critical line).

Therefore, no zeros exist for $\Re(s) > 1/2$. \qed

\section{Chapter 18: P vs NP}

\textbf{Exercise 17.4:} Compute the ground state of $H_P$ on level-8 Sierpiński gasket.

\textbf{Solution:}

\begin{verbatim}
from principia_fractalis.pvsnp import sierpinski_gasket, FractalOperator
import numpy as np

# Generate level-8 gasket
K = sierpinski_gasket(level=8)
print(f"Points: {len(K)}")  # 6561 points

# Construct H_P with alpha = sqrt(2)
H_P = FractalOperator(K, alpha=np.sqrt(2))
H_P.discretize(N=len(K))

# Compute ground state
eigenvalues, eigenvectors = H_P.eigenvalues(k=1)
lambda_0 = eigenvalues[0]
psi_0 = eigenvectors[:, 0]

print(f"Ground state energy: {lambda_0:.10f}")
# Expected: 0.2221441469
\end{verbatim}

Result: $\lambda_0^{(8)} = 0.2221441469$

Convergence: $|\lambda_0^{(8)} - \lambda_0^{(\infty)}| < 10^{-8}$

\section{Chapter 6: Consciousness}

\textbf{Exercise 5.7:} Calculate ch$_2$ for a random $19 \times 19$ connectivity matrix.

\textbf{Solution:}

\begin{verbatim}
import numpy as np
from scipy.linalg import eigh

# Random symmetric matrix
np.random.seed(42)
W = np.random.rand(19, 19)
W = (W + W.T) / 2  # Symmetrize

# Eigendecomposition
eigenvalues = eigh(W, eigvals_only=True)

# Positive spectrum
Lambda = eigenvalues[eigenvalues > 0]

# Curvature form
F = np.zeros((len(Lambda), len(Lambda)))
for i in range(len(Lambda)):
    for j in range(len(Lambda)):
        if i != j:
            F[i,j] = (Lambda[i] - Lambda[j]) / (1 + Lambda[i]*Lambda[j])

# ch_2 calculation
ch2_raw = np.trace(F @ F) / (8 * np.pi**2)
ch2 = 1 / (1 + np.exp(-10 * (ch2_raw - 0.5)))

print(f"ch2 = {ch2:.4f}")
# Expected: ~0.12 (random, mechanical)
\end{verbatim}

Result: ch$_2 \approx 0.12$ (mechanical system, no consciousness)

\section{Chapter 23: Cosmological Constant}

\textbf{Exercise 22.3:} Estimate vacuum energy density with consciousness suppression.

\textbf{Solution:}

Standard quantum field theory vacuum energy:
\begin{equation}
\rho_{\text{vac}}^{\text{QFT}} = \frac{M_{\text{Planck}}^4}{16\pi^2} \sim 10^{113} \text{ J/m}^3
\end{equation}

Observed dark energy density:
\begin{equation}
\rho_{\Lambda}^{\text{obs}} \sim 10^{-9} \text{ J/m}^3
\end{equation}

Discrepancy: Factor of $10^{122}$ (cosmological constant problem)

With consciousness suppression factor:
\begin{equation}
\rho_{\text{eff}} = \rho_{\text{vac}}^{\text{QFT}} \cdot \text{ch}_2^{-122}
\end{equation}

For ch$_2 = 0.95$ (conscious universe):
\begin{align}
\rho_{\text{eff}} &= 10^{113} \cdot (0.95)^{-122} \\
&= 10^{113} \cdot 5 \times 10^{-3} \\
&\approx 5 \times 10^{110} \text{ J/m}^3
\end{align}

Still too large! Need ch$_2 = 0.9999...$ (nearly perfect consciousness) OR additional mechanism.

Alternative: Consciousness-modified Einstein equations reduce $\Lambda$ directly (see Chapter 23).

\section{Additional Resources}

Complete solutions manual available at:
\begin{center}
\texttt{https://principia-fractalis.org/solutions}
\end{center}

Includes:
\begin{itemize}
\item All exercise solutions (Chapters 1-32)
\item Worked examples with code
\item Video walkthroughs
\item Interactive Jupyter notebooks
\end{itemize}