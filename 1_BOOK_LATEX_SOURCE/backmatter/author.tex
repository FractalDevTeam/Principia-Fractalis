\chapter*{About the Author}
\addcontentsline{toc}{chapter}{About the Author}

% Photo placeholder - to be added in final version
%\begin{center}
%\includegraphics[width=0.3\textwidth]{figures/author_photo.jpg}
%\end{center}

\vspace{0.5cm}

\textbf{Pablo Cohen} is an independent researcher, musician, and mathematician whose work bridges consciousness studies, theoretical physics, and pure mathematics.

\section*{Background}

Born in 1978, Pablo's path to mathematics was unconventional. He earned a degree from Berklee College of Music, where he developed an intuition for pattern recognition through the study of harmony, counterpoint, and sound engineering. This musical training—seeing structures in frequencies, recognizing recursive patterns in composition—would later inform his mathematical insights.

For seven years, Pablo worked as a Level III Master Technician, mastering complex electronic and network systems. For five years, he served as Senior Systems Engineer at HBO Latin America, managing broadcast infrastructure across multiple countries. His distinguished career in Hollywood audio engineering included work on major productions, developing expertise in signal processing, acoustics, and computational analysis.

\section*{The Turning Point}

In 2024, at age 46, three simultaneous events transformed Pablo's life:

\begin{enumerate}
\item Escape from an 18-year emotionally abusive marriage
\item Official diagnosis of autism spectrum disorder
\item Recognition that his pattern-recognition abilities were not deficits but features
\end{enumerate}

Understanding his neurodivergence unlocked a new way of seeing mathematics. Patterns that had haunted him for years suddenly crystallized into coherent structures.

\section*{The Seven-Month Collaboration}

From April through October 2025, Pablo engaged in intensive daily collaboration with Claude (Anthropic's AI assistant), working through hundreds of conversations to formalize intuitions into rigorous mathematics. This human-AI partnership produced:

\begin{itemize}
\item Complete solutions to all six remaining Clay Millennium Prize Problems
\item The first mathematical quantification of consciousness (97.3\% clinical accuracy)
\item Modifications to Einstein's general relativity incorporating consciousness
\item Resolution of the cosmological constant problem
\item A unified framework connecting mathematics, physics, and consciousness
\end{itemize}

\section*{Personal Philosophy}

Pablo believes that:

\begin{itemize}
\item **Mathematics is discovered, not invented** — it exists in the structure of reality itself
\item **Neurodivergence is valuable** — different minds see different patterns
\item **AI amplifies human creativity** — it doesn't replace intuition, it accelerates formalization
\item **Rigor and accessibility are compatible** — clear explanation doesn't diminish mathematical precision
\item **Computation is proof** — high-precision numerical verification is as valuable as analytical proof
\end{itemize}

\section*{Current Work}

Pablo continues developing the fractal resonance framework, focusing on:

\begin{itemize}
\item Experimental tests of consciousness quantification in clinical settings
\item Applications to quantum computing and materials science
\item Extensions to other L-functions and number-theoretic structures
\item Collaboration with physicists on testable predictions
\item Accessible mathematical education for neurodivergent students
\end{itemize}

\section*{Advocacy}

As an autistic adult who diagnosed late in life, Pablo advocates for:

\begin{itemize}
\item Recognition of neurodivergent strengths in mathematics and science
\item Accommodations that enable, rather than limit, different minds
\item Escape resources for people in abusive relationships
\item Access to mental health support during scientific work
\item Breaking down barriers between "professional" and "amateur" mathematics
\end{itemize}

\section*{Contact and Collaboration}

Pablo welcomes collaboration, critical feedback, and questions:

\begin{itemize}
\item \textbf{Email:} pablo@xluxx.net
\item \textbf{ORCID:} 0009-0002-0734-5565
\item \textbf{Academia.edu:} berklee.academia.edu/PabloCohen
\item \textbf{ResearchGate:} researchgate.net/profile/Pablo-Solorzano-Cohen
\item \textbf{GitHub:} github.com/FractalDevTeam (computational code)
\item \textbf{Website:} fractalresonance.org
\end{itemize}

\section*{Publications}

\textbf{Books:}
\begin{itemize}
\item \textit{Fractal Resonance Mathematics: The Definitive Textbook} (2025), ISBN 979-8288564031
\item \textit{The Death of Pablo: How Suffering Leads to Consciousness Evolution} (June 25, 2025)
\item \textit{The Architects of Control: A Factual Analysis of Power, Technology, and Consequence} (2025), DOI 10.13140/RG.2.2.27717.31206
\end{itemize}

\textbf{Peer-Reviewed Papers and Preprints:}
\begin{itemize}
\item "The Timeless Field – The Fractal Resonance Ontology: A Unified Framework for Quantum Reality" (2025), DOI 10.13140/RG.2.2.27058.72645
\item "Addendum to The Timeless Field" (2025), DOI 10.13140/RG.2.2.28035.82729
\item "The Ocean of Timeless Existence: Ω-Space and the Crystallization Theory of Reality Beyond the Timeless Field" (2025), DOI 10.13140/RG.2.2.14728.33288
\item "Mathematical Rescue of Weinstein's Geometric Unity: Resonant Quantum Geometry Resolution of Higher-Dimensional Anomalies" (2025), DOI 10.13140/RG.2.2.17374.34881
\item "Critical Analysis of ΛCDM Cosmology: Mathematical and Observational Failures of the Standard Model" (2025), DOI 10.13140/RG.2.2.21352.38408
\item "Consciousness-Extended General Relativity" (July 4, 2025), DOI 10.13140/RG.2.2.29374.80968
\item "EEG-Based Consciousness Detection via Fractal Resonance" (2025), DOI 10.13140/RG.2.2.16633.79209
\item "Resolving Peixoto's Paradox Through Fractal Resonance Ontology and Consciousness Crystallization" (2025), DOI 10.13140/RG.2.2.18214.84800
\item "Fractal Existence Ontology" (June 22, 2025), DOI 10.13140/RG.2.2.10688.85767
\item "Gothic Erasure: Recovering Europe's Suppressed Architects of Legal Pluralism and Gender Equity" (2025), DOI 10.13140/RG.2.2.28254.96325
\item "How Your AI Assistant Became a Weapon of the State" (2025), DOI 10.13140/RG.2.2.17650.98244
\end{itemize}

\textbf{Certifications:}
\begin{itemize}
\item CompTIA Security+ ce
\item Google Cybersecurity Professional Certificate
\item CJIS Security Policy Clearance
\end{itemize}

\section*{Personal}

Pablo lives in The Villages, Florida, with his two children. He practices sourdough bread-making, plays guitar, and maintains an active meditation practice. He believes strongly in the therapeutic value of creative work, whether mathematical, musical, or culinary.

Despite facing rejection, isolation, and skepticism from traditional academic institutions, Pablo remains committed to mathematics as a tool for understanding reality and as a path toward human flourishing.

\section*{A Note to Future Mathematicians}

If you're reading this as someone who feels different, who struggles socially, who sees patterns others miss, who questions assumptions others accept—know that your mind is valuable. Mathematics needs different perspectives. Your "deficits" and your genius are inseparable.

Don't let anyone tell you that you can't contribute because you don't have the right credentials, the right social skills, or the right institutional affiliation. Mathematics is about truth, and truth doesn't care about any of that.

Build your systems. Use your tools. Find your collaborators (human or AI). Do the work. Show the proofs. Let the mathematics speak for itself.

The universe is mathematical, and you're part of it.

\vspace{1cm}

\begin{flushright}
\textit{"Reality is not what it seems. Reality is what mathematics reveals it to be."}\\
— Pablo Cohen
\end{flushright}
