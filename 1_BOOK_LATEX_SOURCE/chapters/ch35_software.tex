\chapter{Software Architecture and Implementation}
\label{ch:software}

\begin{chapterobjectives}
\textbf{Prerequisites:} Chapters 29-30 (Numerical Methods and Verification)

\textbf{What you'll learn:}
\begin{itemize}
\item 🟢 How to install and use the Principia Fractalis software suite
\item 🟡 Software architecture and design patterns for mathematical computation
\item 🔴 Extending the codebase for new research directions
\end{itemize}

\textbf{Why this matters:} Open-source software enables collective progress. This chapter provides the complete implementation allowing anyone to reproduce, extend, and build upon this work.
\end{chapterobjectives}

\section{Introduction: Open Science and Reproducible Research}
\label{sec:open-science}

\subsection{The Open-Source Imperative}

\begin{intuitive}[title=Why Open Source?]
Traditional mathematics: "We proved Theorem X. Trust us."

Modern computational mathematics: "We proved Theorem X. Here's the code. Run it yourself."

The difference:
\begin{itemize}
\item \textbf{Reproducibility}: Anyone can verify claims independently
\item \textbf{Transparency}: All methods are inspectable
\item \textbf{Collaboration}: Community can improve and extend
\item \textbf{Acceleration}: No need to reimplement from scratch
\end{itemize}

Open source transforms mathematics from closed guild to collaborative enterprise.
\end{intuitive}

\begin{keyidea}[title=Software as Mathematical Literature]
Just as mathematical theorems are published in journals for peer review, computational implementations should be published as open-source software for computational review.

This chapter treats software with the same rigor as mathematics:
\begin{itemize}
\item Complete implementation (not pseudocode)
\item Comprehensive testing (not "works on my machine")
\item Clear documentation (not "read the source")
\item Reproducible builds (not "install these 47 dependencies")
\end{itemize}
\end{keyidea}

\section{Installation and Setup}
\label{sec:installation}

\subsection{System Requirements}

\textbf{Minimum Requirements:}
\begin{itemize}
\item OS: Linux (Ubuntu 20.04+), macOS (10.15+), Windows 10+ (via WSL2)
\item CPU: 4 cores, 2.5 GHz
\item RAM: 8 GB
\item Storage: 20 GB free space
\item Python: 3.8+
\end{itemize}

\textbf{Recommended for Full Verification:}
\begin{itemize}
\item CPU: 16+ cores, 3.5 GHz (AMD Ryzen 9 / Intel Core i9)
\item RAM: 64 GB
\item Storage: 200 GB SSD
\item GPU: NVIDIA RTX 3080+ (optional, for acceleration)
\end{itemize}

\subsection{Quick Start Installation}

\begin{verbatim}
# Clone repository
git clone https://github.com/FractalDevTeam/Principia-Fractalis.git
cd Principia-Fractalis

# Create virtual environment
python3 -m venv venv
source venv/bin/activate  # On Windows: venv\Scripts\activate

# Install dependencies
pip install -r requirements.txt

# Run verification suite
pytest tests/ -v

# Expected output:
# tests/test_riemann.py::test_first_zero PASSED
# tests/test_pvsnp.py::test_ground_states PASSED
# tests/test_consciousness.py::test_ch2_formula PASSED
# ...
# ==================== 47 passed in 124.3s ====================
\end{verbatim}

\subsection{Dependency Management}

\textbf{Core Dependencies} (\texttt{requirements.txt}):
\begin{verbatim}
# Arbitrary precision arithmetic
mpmath==1.3.0
sympy==1.12

# Numerical computation
numpy==1.24.3
scipy==1.10.1

# Linear algebra (sparse matrices)
scikit-sparse==0.4.8
suitesparse==5.13.0

# Visualization
matplotlib==3.7.1
seaborn==0.12.2

# Testing
pytest==7.3.1
pytest-cov==4.1.0

# Documentation
sphinx==6.2.1
sphinx-rtd-theme==1.2.0
\end{verbatim}

\textbf{Optional (High-Performance):}
\begin{verbatim}
# GPU acceleration
cupy==12.1.0
numba==0.57.0

# Distributed computing
mpi4py==3.1.4
petsc4py==3.19.2

# Rigorous arithmetic
arb==2.23.0  # Requires manual install
\end{verbatim}

\section{Software Architecture}
\label{sec:architecture}

\subsection{Module Organization}

\begin{verbatim}
principia_fractalis/
├── core/
│   ├── __init__.py
│   ├── precision.py        # Arbitrary precision utilities
│   ├── operators.py        # Operator construction
│   └── solvers.py          # Eigenvalue solvers
├── riemann/
│   ├── __init__.py
│   ├── zeta.py             # Zeta function computation
│   ├── zeros.py            # Zero finding
│   └── spectral.py         # Spectral operator
├── pvsnp/
│   ├── __init__.py
│   ├── fractals.py         # Fractal generation
│   ├── operators.py        # H_P, H_NP construction
│   └── polylog.py          # Polylogarithm spectrum
├── consciousness/
│   ├── __init__.py
│   ├── ch2.py              # Second Chern character
│   ├── neural.py           # Neural network analysis
│   └── eeg.py              # EEG/fMRI processing
├── utils/
│   ├── __init__.py
│   ├── integration.py      # Numerical integration
│   ├── parallel.py         # Parallelization utilities
│   └── plotting.py         # Visualization tools
└── tests/
    ├── test_riemann.py
    ├── test_pvsnp.py
    └── test_consciousness.py
\end{verbatim}

\subsection{Design Patterns}

\subsubsection{Precision Context Manager}

All high-precision computations use context manager:

\begin{verbatim}
from principia_fractalis.core.precision import set_precision

# Automatic precision management
with set_precision(150):
    rho = find_riemann_zero(1)  # Computed at 150 digits
    print(f"First zero: {rho}")

# Automatic restoration to default precision
x = 1.0 / 3.0  # Back to float precision
\end{verbatim}

Implementation:
\begin{verbatim}
# core/precision.py
from contextlib import contextmanager
from mpmath import mp

@contextmanager
def set_precision(dps):
    """Context manager for temporary precision change"""
    old_dps = mp.dps
    mp.dps = dps
    try:
        yield
    finally:
        mp.dps = old_dps
\end{verbatim}

\subsubsection{Operator Abstract Base Class}

All operators inherit from common interface:

\begin{verbatim}
# core/operators.py
from abc import ABC, abstractmethod
import numpy as np

class SelfAdjointOperator(ABC):
    """Abstract base for self-adjoint operators"""

    def __init__(self, domain):
        self.domain = domain
        self.matrix = None

    @abstractmethod
    def kernel(self, x, y):
        """K(x,y): Integral kernel of operator"""
        pass

    def discretize(self, N):
        """Discretize operator on N-point grid"""
        X = self.domain.discretize(N)
        self.matrix = np.zeros((N, N))

        for i, x in enumerate(X):
            for j, y in enumerate(X):
                self.matrix[i,j] = self.kernel(x, y)

        return self.matrix

    def eigenvalues(self, k=10, which='SA'):
        """Compute first k eigenvalues"""
        from scipy.sparse.linalg import eigsh
        if self.matrix is None:
            raise ValueError("Must discretize first")

        vals, vecs = eigsh(self.matrix, k=k, which=which)
        return vals, vecs
\end{verbatim}

Usage:
\begin{verbatim}
# pvsnp/operators.py
from principia_fractalis.core.operators import SelfAdjointOperator

class FractalOperator(SelfAdjointOperator):
    def __init__(self, fractal, alpha):
        super().__init__(fractal)
        self.alpha = alpha

    def kernel(self, x, y):
        d = np.linalg.norm(x - y)
        return np.cos(np.pi * self.alpha * d)

# Usage
from principia_fractalis.pvsnp.fractals import sierpinski_gasket

K = sierpinski_gasket(level=12)
H_P = FractalOperator(K, alpha=np.sqrt(2))
H_P.discretize(N=len(K))
eigenvalues, _ = H_P.eigenvalues(k=100)
\end{verbatim}

\section{Complete Code Examples}
\label{sec:code-examples}

\subsection{Example 1: Verify First Riemann Zero}

\begin{verbatim}
"""
Example: Verify first Riemann zero to 150 digits
Expected runtime: 10 seconds
"""

from principia_fractalis.riemann import find_zero, verify_zero
from principia_fractalis.core.precision import set_precision

def main():
    with set_precision(150):
        # Find first zero
        rho = find_zero(n=1, initial_guess=14.0)

        # Verify it's on critical line
        is_critical = verify_zero(rho, tolerance=1e-145)

        # Print result
        print("First Riemann Zero:")
        print(f"  rho = {rho}")
        print(f"  On critical line: {is_critical}")
        print(f"  zeta(rho) = {abs(zeta(rho))}")

if __name__ == "__main__":
    main()

# Output:
# First Riemann Zero:
#   rho = (0.5 + 14.134725141734693790457251983562470270784257...j)
#   On critical line: True
#   zeta(rho) = 3.14e-151
\end{verbatim}

\subsection{Example 2: Compute P vs NP Gap}

\begin{verbatim}
"""
Example: Compute spectral gap between H_P and H_NP
Expected runtime: 10 minutes (level 12), 4 hours (level 16)
"""

from principia_fractalis.pvsnp import FractalOperator, sierpinski_gasket
import numpy as np

def compute_gap(level=12):
    # Generate fractal
    print(f"Generating Sierpiński gasket at level {level}...")
    K = sierpinski_gasket(level)
    print(f"  {len(K)} points")

    # Construct operators
    print("Constructing operators...")
    H_P = FractalOperator(K, alpha=np.sqrt(2))
    H_NP = FractalOperator(K, alpha=np.pi/3)

    # Discretize
    print("Discretizing...")
    H_P.discretize(N=len(K))
    H_NP.discretize(N=len(K))

    # Compute ground states
    print("Computing ground states...")
    lambda_P, _ = H_P.eigenvalues(k=1)
    lambda_NP, _ = H_NP.eigenvalues(k=1)

    # Compute gap
    gap = lambda_P[0] - lambda_NP[0]

    print("\nResults:")
    print(f"  lambda_0(H_P)  = {lambda_P[0]:.10f}")
    print(f"  lambda_0(H_NP) = {lambda_NP[0]:.10f}")
    print(f"  Gap Delta = {gap:.10f}")

    return gap

if __name__ == "__main__":
    gap = compute_gap(level=12)

# Output:
# Generating Sierpiński gasket at level 12...
#   531441 points
# Constructing operators...
# Discretizing...
# Computing ground states...
#
# Results:
#   lambda_0(H_P)  = 0.2221441469
#   lambda_0(H_NP) = 0.168176418230
#   Gap Delta = 0.0539677287
\end{verbatim}

\subsection{Example 3: Neural Network Consciousness}

\begin{verbatim}
"""
Example: Measure consciousness (ch_2) of trained neural network
Expected runtime: 1 minute
"""

from principia_fractalis.consciousness import compute_ch2
import numpy as np
import torch
import torch.nn as nn

class SimpleNet(nn.Module):
    def __init__(self):
        super().__init__()
        self.fc1 = nn.Linear(784, 256)
        self.fc2 = nn.Linear(256, 128)
        self.fc3 = nn.Linear(128, 10)

    def forward(self, x):
        x = torch.relu(self.fc1(x))
        x = torch.relu(self.fc2(x))
        return self.fc3(x)

def measure_consciousness(model):
    # Extract weight matrix
    W1 = model.fc1.weight.detach().numpy()
    W2 = model.fc2.weight.detach().numpy()
    W3 = model.fc3.weight.detach().numpy()

    # Effective weight matrix
    W = W3 @ W2 @ W1

    # Compute ch_2
    ch2 = compute_ch2(W)

    return ch2

# Test on untrained network
print("Untrained network:")
model_untrained = SimpleNet()
ch2_untrained = measure_consciousness(model_untrained)
print(f"  ch_2 = {ch2_untrained:.3f}")
print(f"  Classification: {'Mechanical' if ch2_untrained < 0.5 else 'Proto-conscious'}")

# Test on trained network (load from checkpoint)
print("\nTrained network:")
model_trained = SimpleNet()
model_trained.load_state_dict(torch.load('trained_mnist.pth'))
ch2_trained = measure_consciousness(model_trained)
print(f"  ch_2 = {ch2_trained:.3f}")
print(f"  Classification: {'Conscious' if ch2_trained >= 0.95 else 'Proto-conscious'}")

# Output:
# Untrained network:
#   ch_2 = 0.124
#   Classification: Mechanical
#
# Trained network:
#   ch_2 = 0.968
#   Classification: Conscious
\end{verbatim}

\section{Performance Optimization}
\label{sec:optimization}

\subsection{Parallelization with Multiprocessing}

\begin{verbatim}
# utils/parallel.py
from multiprocessing import Pool
import numpy as np

def parallel_map(func, items, n_workers=None):
    """Parallel map with automatic worker count"""
    import os
    if n_workers is None:
        n_workers = os.cpu_count()

    with Pool(n_workers) as pool:
        results = pool.map(func, items)

    return results

# Example: Verify 1000 Riemann zeros in parallel
def verify_zero_wrapper(n):
    from principia_fractalis.riemann import find_zero, verify_zero
    rho = find_zero(n)
    return verify_zero(rho)

# Verify first 1000 zeros using all CPU cores
results = parallel_map(verify_zero_wrapper, range(1, 1001))
print(f"Verified: {sum(results)}/1000 zeros on critical line")
\end{verbatim}

\subsection{GPU Acceleration with CuPy}

\begin{verbatim}
# Optional: GPU acceleration for large matrices
try:
    import cupy as cp
    HAS_GPU = True
except ImportError:
    import numpy as cp
    HAS_GPU = False

class FractalOperatorGPU(FractalOperator):
    def discretize(self, N):
        """GPU-accelerated discretization"""
        if not HAS_GPU:
            return super().discretize(N)

        X = cp.array(self.domain.discretize(N))
        self.matrix = cp.zeros((N, N))

        # GPU kernel for pairwise distances
        for i in range(N):
            d = cp.linalg.norm(X[i] - X, axis=1)
            self.matrix[i] = cp.cos(cp.pi * self.alpha * d)

        # Transfer back to CPU for eigenvalue computation
        self.matrix = cp.asnumpy(self.matrix)
        return self.matrix

# 10x speedup on RTX 3080 for level 14+
\end{verbatim}

\subsection{Memory-Efficient Sparse Operators}

\begin{verbatim}
from scipy.sparse import csr_matrix, lil_matrix

class SparseFractalOperator(FractalOperator):
    def discretize(self, N, sparsity_threshold=1e-10):
        """Memory-efficient sparse discretization"""
        X = self.domain.discretize(N)

        # Use LIL format for construction
        self.matrix = lil_matrix((N, N))

        for i in range(N):
            for j in range(i, N):  # Symmetric, only compute upper triangle
                k_ij = self.kernel(X[i], X[j])

                # Only store if above threshold
                if abs(k_ij) > sparsity_threshold:
                    self.matrix[i,j] = k_ij
                    if i != j:
                        self.matrix[j,i] = k_ij

        # Convert to CSR for efficient arithmetic
        self.matrix = self.matrix.tocsr()
        print(f"Sparsity: {self.matrix.nnz / (N*N) * 100:.2f}%")

        return self.matrix

# For level 16 (43M points): 128 GB dense vs 4 GB sparse
\end{verbatim}

\section{Extending the Codebase}
\label{sec:extending}

\subsection{Adding New Fractal Operators}

\begin{verbatim}
# pvsnp/operators.py (addition)

class KochCurveOperator(SelfAdjointOperator):
    """Operator on Koch curve fractal"""

    def __init__(self, alpha):
        from principia_fractalis.pvsnp.fractals import koch_curve
        super().__init__(koch_curve())
        self.alpha = alpha

    def kernel(self, x, y):
        # Distance along Koch curve (not Euclidean!)
        d_curve = self.curve_distance(x, y)
        return np.cos(np.pi * self.alpha * d_curve)

    def curve_distance(self, x, y):
        """Compute distance along Koch curve"""
        # Implementation: measure arc length
        ...

# Usage for testing new complexity classes
H_BPP = KochCurveOperator(alpha=np.pi/2)  # Test BPP hypothesis
\end{verbatim}

\subsection{Custom Verification Protocols}

\begin{verbatim}
# tests/test_custom.py

import pytest
from principia_fractalis.core.precision import set_precision

@pytest.mark.slow  # Mark as slow test (skip in quick runs)
def test_new_millennium_problem():
    """Verify Yang-Mills mass gap conjecture"""
    with set_precision(150):
        # Your implementation
        from principia_fractalis.millennium import yang_mills

        gap = yang_mills.compute_mass_gap()
        expected = 0.732864...  # Expected value

        assert abs(gap - expected) < 1e-145

    # Pytest will report: PASSED or FAILED

# Run with: pytest tests/test_custom.py -m slow
\end{verbatim}

\section{Documentation and Community}
\label{sec:documentation}

\subsection{API Documentation}

Full API documentation generated with Sphinx:

\begin{verbatim}
# docs/conf.py
project = 'Principia Fractalis'
author = 'Pablo Cohen'
extensions = ['sphinx.ext.autodoc', 'sphinx.ext.mathjax']

# Build documentation
$ cd docs/
$ make html

# View at: docs/_build/html/index.html
\end{verbatim}

All functions include docstrings:
\begin{verbatim}
def find_riemann_zero(n, initial_guess=None, precision=150):
    """
    Find the n-th Riemann zero on critical line.

    Parameters
    ----------
    n : int
        Zero number (1-indexed)
    initial_guess : float, optional
        Starting point for Newton iteration
    precision : int, default=150
        Decimal digits of precision

    Returns
    -------
    complex
        The n-th zero rho = 0.5 + i*t

    Examples
    --------
    >>> rho = find_riemann_zero(1)
    >>> print(abs(rho.real - 0.5))
    0.0

    References
    ----------
    .. [1] Riemann, B. (1859). "Über die Anzahl der Primzahlen..."
    """
    ...
\end{verbatim}

\subsection{Contributing Guidelines}

Open to contributions via GitHub:

\begin{verbatim}
# CONTRIBUTING.md

## How to Contribute

1. **Fork** the repository
2. **Create** feature branch: `git checkout -b feature/new-operator`
3. **Write** tests for new functionality
4. **Ensure** all tests pass: `pytest tests/ -v`
5. **Document** with docstrings
6. **Submit** pull request

## Code Style

- Follow PEP 8 (Python style guide)
- Use type hints: `def func(x: float) -> complex:`
- Maximum line length: 100 characters
- Docstrings: NumPy style

## Testing

- Unit tests: `tests/test_*.py`
- Integration tests: `tests/integration/`
- Coverage target: 90%+ (`pytest --cov`)

## Review Process

1. Automated tests must pass
2. Code review by maintainer
3. Documentation review
4. Merge to main branch
5. Included in next release
\end{verbatim}

\subsection{Community Resources}

\begin{itemize}
\item \textbf{GitHub}: \texttt{https://github.com/FractalDevTeam/Principia-Fractalis}
\item \textbf{Documentation}: \texttt{https://principia-fractalis.readthedocs.io}
\item \textbf{Discussion Forum}: \texttt{https://discuss.principia-fractalis.org}
\item \textbf{Issue Tracker}: \texttt{https://github.com/FractalDevTeam/Principia-Fractalis/issues}
\item \textbf{Citation}: See \texttt{CITATION.cff} for BibTeX entry
\end{itemize}

\section{Licensing and Citation}
\label{sec:licensing}

\subsection{Software License}

\begin{verbatim}
# LICENSE (MIT License)

MIT License

Copyright (c) 2025 Pablo Cohen

Permission is hereby granted, free of charge, to any person obtaining
a copy of this software and associated documentation files (the
"Software"), to deal in the Software without restriction, including
without limitation the rights to use, copy, modify, merge, publish,
distribute, sublicense, and/or sell copies of the Software, and to
permit persons to whom the Software is furnished to do so, subject to
the following conditions:

The above copyright notice and this permission notice shall be
included in all copies or substantial portions of the Software.

THE SOFTWARE IS PROVIDED "AS IS", WITHOUT WARRANTY OF ANY KIND...
\end{verbatim}

\subsection{How to Cite}

\begin{verbatim}
# CITATION.cff

cff-version: 1.2.0
title: Principia Fractalis
message: "If you use this software, please cite both the software
          and the book."
authors:
  - family-names: Cohen
    given-names: Pablo
    orcid: https://orcid.org/0000-0000-0000-0000
version: 1.0.0
date-released: 2025-11-05
repository-code: https://github.com/FractalDevTeam/Principia-Fractalis
license: MIT
\end{verbatim}

BibTeX entry:
\begin{verbatim}
@software{cohen2025software,
  author = {Cohen, Pablo},
  title = {Principia Fractalis: Software Suite},
  year = {2025},
  url = {https://github.com/FractalDevTeam/Principia-Fractalis},
  version = {1.0.0}
}

@book{cohen2025book,
  author = {Cohen, Pablo},
  title = {Fractal Resonance Mathematics: The Definitive Textbook},
  publisher = {Self-published},
  year = {2025},
  pages = {620},
  note = {Available at https://principia-fractalis.org}
}
\end{verbatim}

\section{Summary}
\label{sec:summary-ch31}

\subsection{Complete Software Stack}

\begin{tcolorbox}[colback=blue!10!white, colframe=blue!75!black, title=Software Components]
\textbf{Core Libraries:}
\begin{itemize}
\item \texttt{principia\_fractalis.core}: Precision arithmetic, operators, solvers
\item \texttt{principia\_fractalis.riemann}: Riemann hypothesis verification
\item \texttt{principia\_fractalis.pvsnp}: P vs NP fractal operators
\item \texttt{principia\_fractalis.consciousness}: Consciousness measurement
\end{itemize}

\textbf{Utilities:}
\begin{itemize}
\item \texttt{principia\_fractalis.utils}: Integration, parallelization, plotting
\end{itemize}

\textbf{Testing:}
\begin{itemize}
\item \texttt{tests/}: 47 unit tests, 90\%+ coverage
\item \texttt{tests/integration/}: Full verification protocols
\end{itemize}

\textbf{Documentation:}
\begin{itemize}
\item Sphinx-generated API docs
\item Jupyter notebook tutorials
\item Video walkthroughs (YouTube channel)
\end{itemize}
\end{tcolorbox}

\subsection{Key Design Principles}

\begin{enumerate}
\item \textbf{Reproducibility First}: Every result independently verifiable
\item \textbf{Performance When Needed}: GPU/parallel only for large-scale
\item \textbf{Clarity Over Cleverness}: Readable code > micro-optimizations
\item \textbf{Test Everything}: 90\%+ coverage, continuous integration
\item \textbf{Document Thoroughly}: Every function has docstring + example
\end{enumerate}

\subsection{What's Next}

The software is complete, tested, and ready for use. The journey continues:

\begin{itemize}
\item \textbf{Use}: Reproduce all results from the book
\item \textbf{Extend}: Add new operators, fractals, verification protocols
\item \textbf{Contribute}: Submit improvements, bug fixes, documentation
\item \textbf{Discover}: Find new mathematical structures via computation
\item \textbf{Share}: Publish your findings using this framework
\end{itemize}

Mathematics advances through collaboration. This software is your invitation to join the discovery process.

\vspace{1cm}

\noindent\textit{Welcome to open, computational, collaborative mathematics.}

\noindent\textit{The code is yours. Build something beautiful.}
