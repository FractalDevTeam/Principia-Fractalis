\chapter{Complex Analysis Fundamentals}
\label{ch:complex}

\begin{chapterobjectives}
\textbf{Prerequisites:} Chapter 1 (Numbers and Base-3 Arithmetic), multivariable calculus

\textbf{What you'll learn:}
\begin{itemize}
\item 🟢 Analytic continuation and monodromy along paths
\item 🟡 Branch structure, multivalued functions, and winding
\item 🔴 Polylogarithms and singular expansions near $z=1$
\end{itemize}

\textbf{Why this matters:} This chapter establishes the exact analytic framework used in all downstream proofs, particularly the P vs NP monodromy arguments (Chapter~\ref{ch:p-vs-np}) and Riemann Hypothesis (Chapter~\ref{ch:riemann-hypothesis}). Every definition, theorem, and label here is referenced later. The key insight: nonlinearity under winding ($w \mapsto w + 2\pi i m$ for $s \notin \Z$) is what distinguishes P from NP.
\end{chapterobjectives}

\section{Preliminaries and Notation}
\label{sec:preliminaries}

We establish conventions used throughout this volume.

\subsection{Domains and Simple Connectivity}

\begin{definition}[title=Domain]\index{domain}
A \textbf{domain} $U \subset \C$ is a connected open subset of the complex plane.
\end{definition}

\begin{definition}[title=Simply Connected]\index{simply connected}
A domain $U$ is \textbf{simply connected} if every closed curve in $U$ can be continuously contracted to a point within $U$. Equivalently, $U$ has no "holes."
\end{definition}

\begin{example}[title=Simply Connected Domains]
\begin{itemize}
\item $\C$ (the entire complex plane): simply connected
\item $\C \setminus \{0\}$ (plane with origin removed): NOT simply connected (has a hole)
\item $\C \setminus (-\infty, 0]$ (plane with negative real axis removed): simply connected
\end{itemize}
\end{example}

\subsection{Disk Notation}

For $z_0 \in \C$ and $r > 0$:
\begin{align}
B(z_0, r) &:= \{z \in \C : |z - z_0| < r\} \quad \text{(open disk)} \\
\overline{B}(z_0, r) &:= \{z \in \C : |z - z_0| \leq r\} \quad \text{(closed disk)}
\end{align}

\subsection{Principal Branch of Logarithm and Argument}

\begin{definition}[title=Principal Logarithm]\index{principal logarithm}\label{def:principal-log}
Define the \textbf{principal branch} of the complex logarithm $\Log: \C \setminus (-\infty, 0] \to \C$ by:
\begin{equation}
\Log(r e^{i\theta}) = \log r + i\theta
\end{equation}
where $r > 0$ and $\theta \in (-\pi, \pi]$. Here $\log r$ denotes the real natural logarithm.
\end{definition}

\begin{definition}[title=Principal Argument]\index{principal argument}
The \textbf{principal argument} $\Arg: \C \setminus \{0\} \to (-\pi, \pi]$ is the unique angle $\theta \in (-\pi, \pi]$ such that $z = |z| e^{i\theta}$.
\end{definition}

\begin{remark}[Branch Cut Convention]
The principal branch $\Log$ has a \textbf{branch cut}\index{branch cut} along the negative real axis $(-\infty, 0]$. Crossing this cut jumps $\Log z$ by $\pm 2\pi i$.
\end{remark}

\subsection{Holomorphic and Meromorphic Functions}

\begin{definition}[title=Holomorphic Function]\index{holomorphic function}
A function $f: U \to \C$ (where $U \subseteq \C$ is open) is \textbf{holomorphic} if it is complex differentiable at every point in $U$. That is, for each $z_0 \in U$, the limit
\begin{equation}
f'(z_0) = \lim_{h \to 0} \frac{f(z_0 + h) - f(z_0)}{h}
\end{equation}
exists, where $h \in \C$ can approach $0$ from any direction.
\end{definition}

\begin{definition}[title=Meromorphic Function]\index{meromorphic function}
A function $f: U \to \C$ is \textbf{meromorphic} if it is holomorphic except at a set of isolated poles.
\end{definition}

\begin{definition}[title=Isolated Singularity]\index{isolated singularity}
A point $z_0$ is an \textbf{isolated singularity} of $f$ if $f$ is holomorphic in some punctured disk $0 < |z - z_0| < \epsilon$ but not at $z_0$ itself.
\end{definition}

\section{Integral Foundations}
\label{sec:integral-foundations}

\subsection{Cauchy–Goursat Theorem}

\begin{theorem}[title=Cauchy–Goursat]\index{Cauchy-Goursat theorem}\label{thm:cauchy-goursat}
Let $f$ be holomorphic in a simply connected domain $U$, and let $\gamma$ be a closed contour in $U$. Then:
\begin{equation}
\int_{\gamma} f(z) \, dz = 0
\end{equation}
\end{theorem}

\begin{intuitive}[title=Why Holomorphic Integrals Vanish]
Holomorphic functions behave like "smooth flows" with no circulation. Integrating around a closed loop in a simply connected region yields zero—you return to where you started with no net accumulation.
\end{intuitive}

\subsection{Cauchy Integral Formula and Higher Derivatives}

\begin{theorem}[title=Cauchy Integral Formula (CIF)]\index{Cauchy integral formula}\label{thm:CIF}
If $f$ is holomorphic on an open set containing $\overline{B(z_0, R)}$ and $\gamma$ is the positively oriented boundary circle $|z - z_0| = R$, then for $z$ in the interior:
\begin{equation}
f(z) = \frac{1}{2\pi i} \int_{\gamma} \frac{f(\zeta)}{\zeta - z} \, d\zeta
\end{equation}
\end{theorem}

\begin{corollary}[Higher Derivatives]\label{cor:CIF-derivatives}
Under the same hypotheses:
\begin{equation}
f^{(n)}(z) = \frac{n!}{2\pi i} \int_{\gamma} \frac{f(\zeta)}{(\zeta - z)^{n+1}} \, d\zeta
\end{equation}
\end{corollary}

\begin{keyidea}[title=Rigidity of Holomorphic Functions]
The CIF implies: knowing $f$ on any circle determines $f$ and \textit{all its derivatives} at the center. This rigidity is essential—holomorphic functions are essentially unique once specified on any open set.
\end{keyidea}

\subsection{Morera's Theorem}

\begin{theorem}[title=Morera]\index{Morera's theorem}\label{thm:morera}
If $f$ is continuous on a domain $U$ and $\int_{\gamma} f(\zeta) \, d\zeta = 0$ for all triangles $\gamma \subset U$, then $f$ is holomorphic on $U$.
\end{theorem}

\subsection{Liouville's Theorem}

\begin{theorem}[title=Liouville]\index{Liouville's theorem}\label{thm:liouville}
If $f$ is holomorphic on all of $\C$ (entire) and bounded, then $f$ is constant.
\end{theorem}

\subsection{Maximum Modulus Principle}

\begin{theorem}[title=Maximum Modulus]\index{maximum modulus principle}\label{thm:maximum-modulus}
If $f$ is holomorphic and non-constant on a domain $U$, then $|f|$ cannot attain a local maximum in the interior of $U$.
\end{theorem}

\subsection{Schwarz Lemma}

\begin{lemma}[Schwarz]\index{Schwarz lemma}\label{lem:schwarz}
If $f: B(0,1) \to B(0,1)$ is holomorphic with $f(0) = 0$, then:
\begin{enumerate}
\item $|f(z)| \leq |z|$ for all $z \in B(0,1)$
\item $|f'(0)| \leq 1$
\item If $|f(z)| = |z|$ for some $z \neq 0$, or if $|f'(0)| = 1$, then $f(z) = e^{i\theta} z$ for some $\theta \in \R$.
\end{enumerate}
\end{lemma}

\subsection{Identity Theorem}

\begin{theorem}[title=Identity Theorem]\index{identity theorem}\label{thm:identity}
If $f, g$ are holomorphic on a connected domain $U$ and agree on a set with an accumulation point in $U$, then $f \equiv g$ on $U$.
\end{theorem}

\section{Analytic Continuation and Monodromy}
\label{sec:analytic-continuation}

This section establishes the framework for handling multivalued functions—essential for understanding the nonlinearity that separates P from NP.

\subsection{Germs and Analytic Continuation Along Paths}

\begin{definition}[title=Germ]\index{germ}\label{def:germ}
Let $f$ be holomorphic on a domain $U \subset \C$ and $z_0 \in U$. A \textbf{germ} at $z_0$ is an equivalence class of pairs $(V, g)$ where $z_0 \in V \subset U$ is open, $g$ is holomorphic on $V$, and two pairs $(V_1, g_1)$, $(V_2, g_2)$ are equivalent if $g_1 = g_2$ on some neighborhood of $z_0$.
\end{definition}

\begin{definition}[title=Analytic Continuation Along a Path]\index{analytic continuation}\label{def:analytic-continuation}
Given a continuous path $\gamma: [0,1] \to \C$ and a germ of $f$ at $\gamma(0)$, an \textbf{analytic continuation of $f$ along $\gamma$} is a family of germs $\{(V_t, f_t)\}_{t \in [0,1]}$ such that:
\begin{enumerate}
\item $f_0$ agrees with the germ of $f$ at $\gamma(0)$
\item For each $t \in [0,1]$, $(V_t, f_t)$ is a germ at $\gamma(t)$
\item For $s, t$ sufficiently close, $f_s$ and $f_t$ agree on a neighborhood of $\gamma([s, t])$
\end{enumerate}
\end{definition}

\subsection{Monodromy Theorem}

\begin{theorem}[title=Monodromy Theorem]\index{monodromy theorem}\label{thm:monodromy}
Let $U \subset \C$ be simply connected. Suppose $f$ admits analytic continuation along any path in $U$ starting from $z_0 \in U$. Then the continuation is single-valued on $U$; i.e., the value of $f$ at each $z \in U$ is independent of the path from $z_0$ to $z$.
\end{theorem}

\begin{proof}[Proof Sketch]
In a simply connected domain, any two paths with the same endpoints are homotopic. Analytic continuation is invariant under homotopy, so the final germ depends only on the endpoint, not the path.
\end{proof}

\subsection{Multivalued Functions and Riemann Surfaces}

\begin{remark}[Multivalued Functions]\index{multivalued function}
When $U$ is \textit{not} simply connected (e.g., $\C \setminus \{0\}$), analytic continuation along different paths can yield different values. We call such a function \textbf{multivalued}.

Formally, multivalued functions are single-valued on their \textbf{Riemann surface}\index{Riemann surface}—a covering space where each "sheet" corresponds to a branch.
\end{remark}

\subsection{Branch Cuts and Branches}

\begin{definition}[title=Branch]\index{branch}
A \textbf{branch} of a multivalued function is a single-valued holomorphic function defined on a domain obtained by removing a branch cut.
\end{definition}

\begin{example}[title=Principal Branch of $\Log$]
The principal logarithm $\Log$ is a branch of the multivalued logarithm, defined on $\C \setminus (-\infty, 0]$ with branch cut along the negative real axis.
\end{example}

\subsection{Small Loops Around Branch Points}

\begin{remark}[Precise Language for Winding]\index{winding}
When we say "a small loop around a branch point," we mean a continuous closed path $\gamma: [0,1] \to \C$ with $\gamma(0) = \gamma(1)$, encircling the branch point once positively (counterclockwise). Analytically continuing a function along such a loop reveals the monodromy action.
\end{remark}

\section{Principal Logarithm, Fractional Powers, and Winding}
\label{sec:logarithm-fractional-powers}

This section provides the \textit{exact} nonlinearity used in the P vs NP proof.

\subsection{Principal Branch of Logarithm}

\begin{definition}[title={Principal Branch of $\Log$ (reprise)}]\label{def:Log}
Recall from Definition~\ref{def:principal-log}: $\Log: \C \setminus (-\infty, 0] \to \C$ is defined by:
\begin{equation}
\Log(r e^{i\theta}) = \log r + i\theta, \quad r > 0, \; \theta \in (-\pi, \pi]
\end{equation}
\end{definition}

\subsection{Fractional Powers on the Principal Sheet}

\begin{definition}[title=Fractional Power]\index{fractional power}\label{def:fractional-power}
For $\beta \in \C$ and $z \in \C \setminus (-\infty, 0]$, define:
\begin{equation}
z^{\beta} := \exp\big(\beta \Log z\big)
\end{equation}
This is the \textbf{principal branch} of $z^{\beta}$.
\end{definition}

\subsection{Winding Action on $\Log$}

\begin{lemma}[Winding Action on $\Log$]\index{winding!on logarithm}\label{lem:winding-log}
Let $\gamma$ be a closed loop winding $m \in \Z$ times positively around $0$. Then analytic continuation of $\Log$ along $\gamma$ yields:
\begin{equation}
\Log z \mapsto \Log z + 2\pi i m
\end{equation}
\end{lemma}

\begin{proof}
Each positive winding increases the argument by $2\pi$, so $\Arg z \mapsto \Arg z + 2\pi m$. Since $\Log z = \log |z| + i \Arg z$, we have $\Log z \mapsto \Log z + 2\pi i m$.
\end{proof}

\subsection{Nonlinearity of Fractional Powers Under Winding}

This is the key lemma referenced in Chapter~\ref{ch:p-vs-np}.

\begin{lemma}[Nonlinearity of Fractional Powers Under Winding]\index{fractional power!nonlinearity}\label{lem:frac-nonlinear}
Let $s \in \C \setminus \Z$ and $w \in \C \setminus (-\infty, 0]$. Under analytic continuation corresponding to $w \mapsto w + 2\pi i m$ (where $m \in \Z$), the fractional power transforms as:
\begin{equation}
w^{s-1} \mapsto (w + 2\pi i m)^{s-1} = \sum_{k=0}^{\infty} \binom{s-1}{k} (2\pi i m)^k w^{s-1-k}
\end{equation}
where $\binom{s-1}{k} = \frac{(s-1)(s-2) \cdots (s-k)}{k!}$ is the generalized binomial coefficient.

This is a genuinely \textbf{nonlinear} dependence on $m$: all powers $m^k$ appear for $k \geq 0$.
\end{lemma}

\begin{proof}
By the binomial theorem (extended to complex exponents):
\begin{equation}
(w + 2\pi i m)^{s-1} = w^{s-1} (1 + 2\pi i m / w)^{s-1} = w^{s-1} \sum_{k=0}^{\infty} \binom{s-1}{k} (2\pi i m / w)^k
\end{equation}
Expanding:
\begin{equation}
= \sum_{k=0}^{\infty} \binom{s-1}{k} (2\pi i m)^k w^{s-1-k}
\end{equation}
This series involves all powers $m^0, m^1, m^2, \ldots$, giving nonlinearity in $m$.
\end{proof}

\begin{remark}[Special Case: $s \in \Z$]\label{rem:integer-s}
If $s \in \Z$, then $\binom{s-1}{k} = 0$ for $k \geq s$, so the binomial series truncates. The dependence on $m$ becomes a polynomial of degree $s-1$, not an infinite series. This is why $s \notin \Z$ is crucial for the P vs NP argument.
\end{remark}

\section{Polylogarithms and Jonquières Expansion}
\label{sec:polylogarithms}

The polylogarithm $\Li_s(z)$ is essential for understanding the singularity structure of Dirichlet series near $z = 1$.

\subsection{Definition and Integral Representation}

\begin{definition}[title=Polylogarithm]\index{polylogarithm}\label{def:polylog}
For $|z| < 1$ and $s \in \C$, define:
\begin{equation}
\Li_s(z) := \sum_{n=1}^{\infty} \frac{z^n}{n^s}
\end{equation}
\end{definition}

\begin{proposition}[Integral Representation]\index{polylogarithm!integral representation}\label{prop:polylog-integral}
If $\Re s > 0$ and $z \in \C \setminus [1, \infty)$, then:
\begin{equation}
\Li_s(z) = \frac{z}{\Gamma(s)} \int_0^{\infty} \frac{t^{s-1}}{e^t - z} \, dt
\end{equation}
\end{proposition}

\begin{proof}[Proof Sketch]
Expand $(e^t - z)^{-1} = e^{-t} (1 - z e^{-t})^{-1} = e^{-t} \sum_{n=0}^{\infty} z^n e^{-nt}$ for $|z e^{-t}| < 1$. Substitute into the integral and interchange summation and integration (justified for $\Re s > 0$):
\begin{align}
\frac{z}{\Gamma(s)} \int_0^{\infty} \frac{t^{s-1}}{e^t - z} \, dt &= \frac{z}{\Gamma(s)} \sum_{n=1}^{\infty} z^{n-1} \int_0^{\infty} t^{s-1} e^{-nt} \, dt \\
&= \frac{z}{\Gamma(s)} \sum_{n=1}^{\infty} z^{n-1} \frac{\Gamma(s)}{n^s} = \sum_{n=1}^{\infty} \frac{z^n}{n^s}
\end{align}
\end{proof}

\subsection{Singular Expansion Near $z = 1$}

This expansion is referenced in the P vs NP monodromy analysis.

\begin{theorem}[title={Singular Expansion Near $z = 1$}]\index{polylogarithm!singular expansion}\label{thm:Li-expansion}
Let $w = -\Log z$ on the principal branch, so that $z = e^{-w}$ with $|\arg w| < \pi$. If $s \notin \{1, 2, 3, \ldots\}$, then as $w \to 0$:
\begin{equation}
\Li_s(e^{-w}) = \Gamma(1-s) w^{s-1} + \sum_{k=0}^{\infty} \zeta(s-k) \frac{(-w)^k}{k!}
\end{equation}
where $\zeta$ is the Riemann zeta function and the series converges for $|w| < 2\pi$.
\end{theorem}

\begin{proof}[Proof Sketch (See Zagier, 2007)]
Use the functional equation for $\Li_s$ and expand near the singularity at $z = 1$. The leading singular term comes from the Mellin transform of $\Li_s$, yielding $\Gamma(1-s) w^{s-1}$. The regular part is expressed via zeta values $\zeta(s-k)$.
\end{proof}

\begin{corollary}[Monodromy of $\Li_s$ Around $z = 1$]\index{polylogarithm!monodromy}\label{cor:Li-monodromy}
Continuing once positively around $z = 1$ corresponds to $w \mapsto w + 2\pi i$. By Lemma~\ref{lem:frac-nonlinear}, the jump of $\Li_s$ across the cut involves the full binomial series in $(2\pi i)$ when $s \notin \Z$:
\begin{equation}
\Delta \Li_s(e^{-w}) = \Gamma(1-s) \left[ (w + 2\pi i)^{s-1} - w^{s-1} \right]
\end{equation}
This is nonlinear in the winding number, which is the basis for distinguishing P from NP complexity classes.
\end{corollary}

\subsection{Logarithmic Case: $s \in \mathbb{N}$}

\begin{remark}[Special Case: $s \in \mathbb{N}$]\label{rem:Li-integer}
When $s \in \mathbb{N}$, the term $\Gamma(1-s)$ has a pole, and the expansion includes logarithmic terms:
\begin{equation}
\Li_s(e^{-w}) = \frac{(-1)^{s-1}}{(s-1)!} (\log w)^{s-1} + \text{(regular terms)}
\end{equation}
This is why the nonlinearity argument requires $s \notin \Z$.
\end{remark}

\subsection{Jonquières-Type Expansion (Statement)}

\begin{theorem}[title=Jonquières Expansion, Statement]\index{Jonquières expansion}\label{thm:jonquieres}
For $|z-1| < 1$ and $z \neq 1$, the polylogarithm $\Li_s(z)$ admits an expansion of the form:
\begin{equation}
\Li_s(z) = \sum_{k=0}^{\infty} a_k(s) (z-1)^k + \text{(singular terms)}
\end{equation}
where the coefficients $a_k(s)$ involve zeta values and the singular terms depend on the chosen branch.
\end{theorem}

\begin{remark}[Deferred Proof]
The full proof of the Jonquières expansion is deferred to Chapter~\ref{ch:p-vs-np}, where it is used "at scale" to analyze the monodromy of $R_f(\alpha, s)$ near critical points.
\end{remark}

\section{Convergence, Uniformity, and Interchange of Limits}
\label{sec:dirichlet-convergence}

\subsection{Uniform Convergence and Termwise Operations}

\begin{lemma}[Uniform Convergence Window]\index{uniform convergence}\label{lem:uniform-window}
Let $\sum_{n=1}^{\infty} f_n$ be a series of holomorphic functions on a domain $U$, and suppose the series converges uniformly on every compact subset $K \Subset U$ (i.e., $K$ is a compact subset of $U$). Then:
\begin{enumerate}
\item $\sum_{n=1}^{\infty} f_n$ is holomorphic on $U$
\item Termwise differentiation holds: $\displaystyle \left( \sum_{n=1}^{\infty} f_n \right)' = \sum_{n=1}^{\infty} f_n'$ on $U$
\item Termwise integration holds: $\displaystyle \int_{\gamma} \sum_{n=1}^{\infty} f_n = \sum_{n=1}^{\infty} \int_{\gamma} f_n$ for any contour $\gamma \subset U$
\end{enumerate}
\end{lemma}

\begin{proof}
Uniform convergence on compact sets implies uniform convergence on contours, which justifies interchange of summation and integration by the Weierstrass M-test. Holomorphicity follows from Morera's theorem (Theorem~\ref{thm:morera}). Termwise differentiation follows from Cauchy's formula for derivatives (Corollary~\ref{cor:CIF-derivatives}).
\end{proof}

\subsection{Abel's Theorem}

\begin{theorem}[title=Abel's Theorem]\index{Abel's theorem}\label{thm:abel}
Let $\sum_{n=0}^{\infty} a_n z^n$ be a power series with radius of convergence $R = 1$, and suppose $\sum_{n=0}^{\infty} a_n$ converges. Then:
\begin{equation}
\lim_{r \uparrow 1} \sum_{n=0}^{\infty} a_n r^n = \sum_{n=0}^{\infty} a_n
\end{equation}
That is, the power series converges to the sum of its coefficients as $r \to 1^-$ along the real axis.
\end{theorem}

\begin{proof}[Proof Sketch]
Use summation by parts to control the tail of the series, combined with the fact that $r^n \to 1$ as $r \uparrow 1$.
\end{proof}

\begin{remark}[Application to Dirichlet Series]
Abel's theorem is used extensively in analytic number theory to relate Dirichlet series $\sum a_n / n^s$ to their limits as $\Re s \to \sigma_c$ from above, where $\sigma_c$ is the abscissa of convergence.
\end{remark}

\section{Cross-Chapter Dependencies}
\label{sec:dependencies}

This chapter provides the foundation for:

\begin{itemize}
\item \textbf{Chapter~\ref{ch:resonance}:} Analytic continuation of $R_f(\alpha, s)$
\item \textbf{Chapter~\ref{ch:riemann-hypothesis}:} Zeros of $\zeta(s)$ and functional equation
\item \textbf{Chapter~\ref{ch:p-vs-np}:} Monodromy nonlinearity (Lemma~\ref{lem:frac-nonlinear}), polylogarithm expansions (Theorem~\ref{thm:Li-expansion})
\item \textbf{Chapter~\ref{ch:navier-stokes}:} Contour deformation in integral representations
\item \textbf{Chapter~\ref{ch:yang-mills}:} Branch structure of partition functions
\end{itemize}

\textbf{Key Labels for Downstream Reference:}
\begin{itemize}
\item \verb|\label{lem:frac-nonlinear}| — Nonlinearity of $(w + 2\pi i m)^{s-1}$ for $s \notin \Z$
\item \verb|\label{thm:Li-expansion}| — Singular expansion $\Li_s(e^{-w})$ near $w = 0$
\item \verb|\label{cor:Li-monodromy}| — Monodromy jump of $\Li_s$ around $z = 1$
\item \verb|\label{thm:abel}| — Abel's theorem for boundary behavior
\item \verb|\label{lem:uniform-window}| — Uniform convergence and termwise operations
\end{itemize}

\section{Summary and Looking Ahead}

In this chapter, we established the analytic framework for all downstream proofs:

\begin{enumerate}
\item \textbf{Preliminaries:} Domains, simple connectivity, $\Log$ and $\Arg$ conventions
\item \textbf{Integral Foundations:} CIF, Morera, Liouville, Schwarz lemma, identity theorem
\item \textbf{Analytic Continuation:} Germs, path-dependence, monodromy theorem
\item \textbf{Logarithm \& Fractional Powers:} Winding action $w \mapsto w + 2\pi i m$ induces \textbf{nonlinear} monodromy for $s \notin \Z$ (Lemma~\ref{lem:frac-nonlinear})
\item \textbf{Polylogarithms:} Singular expansion near $z = 1$ (Theorem~\ref{thm:Li-expansion}), monodromy jump (Corollary~\ref{cor:Li-monodromy})
\item \textbf{Convergence:} Uniform convergence, Abel's theorem
\end{enumerate}

\textbf{In Chapter 3}, we will:
\begin{itemize}
\item Define $R_f(\alpha, s)$ rigorously using the tools developed here
\item Prove analytic continuation and functional equations
\item Show how different resonance frequencies $\alpha$ address different scientific problems
\item Connect the nonlinearity of winding to the separation of P and NP
\end{itemize}

The key insight: \textit{nonlinearity under winding} (Lemma~\ref{lem:frac-nonlinear}) is not a technical detail—it is the mathematical reason why nondeterministic computation cannot be efficiently simulated by deterministic algorithms.

% End of Chapter 2
