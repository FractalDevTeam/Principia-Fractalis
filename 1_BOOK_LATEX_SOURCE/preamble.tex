% PREAMBLE FOR FRACTAL RESONANCE MATHEMATICS TEXTBOOK
% Author: Pablo Cohen
% ORCID: 0009-0002-0734-5565

% ============================================
% PACKAGES
% ============================================

% Page layout and formatting
\usepackage[letterpaper,margin=1in,includefoot]{geometry}
\usepackage{fancyhdr}
\usepackage{setspace}

% Mathematics
\usepackage{amsmath,amssymb,amsthm}
\usepackage{mathtools}
\usepackage{physics}
\usepackage{tensor}

% Graphics and color
\usepackage{graphicx}
\usepackage[dvipsnames,table]{xcolor}
\usepackage{tikz}
\usetikzlibrary{arrows,shapes,positioning,patterns,calc,backgrounds}
\usepackage{pgfplots}
\pgfplotsset{compat=1.18}
\usepgfplotslibrary{fillbetween}

% Better float placement
\usepackage{float}
\usepackage{placeins}  % Provides \FloatBarrier
\renewcommand{\topfraction}{0.9}
\renewcommand{\bottomfraction}{0.8}
\setcounter{topnumber}{2}
\setcounter{bottomnumber}{2}
\setcounter{totalnumber}{4}
\renewcommand{\textfraction}{0.07}
\renewcommand{\floatpagefraction}{0.7}

% Principia Fractalis color palette
\definecolor{pfblack}{HTML}{000000}
\definecolor{pfgold}{HTML}{F0C419}
\definecolor{pfblue}{HTML}{4AA3FF}
\definecolor{pfwhite}{HTML}{FFFFFF}

% Tables and lists
\usepackage{booktabs}
\usepackage{multirow}
\usepackage{enumitem}

% Code listings
\usepackage{listings}
\usepackage{algorithm}
\usepackage{algorithmic}

% References and citations
\usepackage{url}
\usepackage[numbers,sort&compress]{natbib}  % Scientific bibliography with DOI support
\usepackage{doi}  % Proper DOI formatting and hyperlinking
\usepackage{hyperref}
\usepackage{cleveref}

% Boxes and frames
\usepackage{tcolorbox}
\tcbuselibrary{theorems,skins,breakable}

% Index
\usepackage{makeidx}
\makeindex

% Fonts and Unicode support
\usepackage[T1]{fontenc}
\usepackage[utf8]{inputenc}
\usepackage{lmodern}

% Define emoji replacements for pdflatex
\DeclareUnicodeCharacter{1F7E2}{[L1]} % Green circle
\DeclareUnicodeCharacter{1F7E1}{[L2]} % Yellow circle
\DeclareUnicodeCharacter{1F534}{[L3]} % Red circle
\DeclareUnicodeCharacter{1F4BB}{[CODE]} % Computer
\DeclareUnicodeCharacter{1F4CA}{[GRAPH]} % Bar chart
\DeclareUnicodeCharacter{1F3AF}{[TARGET]} % Target
\DeclareUnicodeCharacter{26A0}{[!]} % Warning sign
\DeclareUnicodeCharacter{039B}{$\Lambda$} % Lambda
\DeclareUnicodeCharacter{03A9}{$\Omega$} % Omega
\DeclareUnicodeCharacter{2713}{[OK]} % Checkmark
\DeclareUnicodeCharacter{221E}{$\infty$} % Infinity
\DeclareUnicodeCharacter{2500}{-} % Box drawing horizontal
\DeclareUnicodeCharacter{2502}{|} % Box drawing vertical
\DeclareUnicodeCharacter{251C}{+} % Box drawing vertical and right
\DeclareUnicodeCharacter{2514}{+} % Box drawing up and right

% ============================================
% COLOR DEFINITIONS
% ============================================

\definecolor{defblue}{RGB}{0,102,204}
\definecolor{thmgreen}{RGB}{34,139,34}
\definecolor{exampleyellow}{RGB}{255,215,0}
\definecolor{warnred}{RGB}{220,20,60}
\definecolor{histpurple}{RGB}{147,112,219}
\definecolor{level1green}{RGB}{144,238,144}
\definecolor{level2yellow}{RGB}{255,255,153}
\definecolor{level3red}{RGB}{255,182,193}

% ============================================
% CUSTOM ENVIRONMENTS - COLOR CODED BOXES
% ============================================

% Blue Box - Definitions
\newtcolorbox{definition}[1][]{
  colback=defblue!5,
  colframe=defblue,
  fonttitle=\bfseries,
  title=Definition,
  breakable,
  enhanced,
  before skip=10pt plus 2pt,
  after skip=10pt plus 2pt,
  #1
}

% Green Box - Theorems
\newtcolorbox{theorem}[1][]{
  colback=thmgreen!5,
  colframe=thmgreen,
  fonttitle=\bfseries,
  title=Theorem,
  breakable,
  enhanced,
  before skip=10pt plus 2pt,
  after skip=10pt plus 2pt,
  #1
}

% Yellow Box - Examples
\newtcolorbox{example}[1][]{
  colback=exampleyellow!10,
  colframe=exampleyellow!80!black,
  fonttitle=\bfseries,
  title=Example,
  breakable,
  enhanced,
  before skip=10pt plus 2pt,
  after skip=10pt plus 2pt,
  #1
}

% Red Box - Warnings
\newtcolorbox{warning}[1][]{
  colback=warnred!5,
  colframe=warnred,
  fonttitle=\bfseries,
  title=Warning,
  breakable,
  enhanced,
  before skip=10pt plus 2pt,
  after skip=10pt plus 2pt,
  #1
}

% Purple Box - Historical Context
\newtcolorbox{historical}[1][]{
  colback=histpurple!5,
  colframe=histpurple,
  fonttitle=\bfseries,
  title=Historical Context,
  breakable,
  enhanced,
  before skip=10pt plus 2pt,
  after skip=10pt plus 2pt,
  #1
}

\newtcolorbox{historicalnote}[1][]{
  colback=histpurple!5,
  colframe=histpurple,
  fonttitle=\bfseries,
  title=Historical Note,
  breakable,
  enhanced,
  #1
}

% Red Box - Warning
\newtcolorbox{warningbox}[1][]{
  colback=warnred!5,
  colframe=warnred,
  fonttitle=\bfseries,
  title=Warning,
  breakable,
  enhanced,
  #1
}

% Light Green Box - Intuitive Understanding
\newtcolorbox{intuitive}[1][]{
  colback=level1green!15,
  colframe=level1green!80!black,
  fonttitle=\bfseries,
  title=Understanding Intuitively,
  breakable,
  enhanced,
  #1
}

% Cyan Box - Key Ideas
\newtcolorbox{keyidea}[1][]{
  colback=cyan!5,
  colframe=cyan!60!black,
  fonttitle=\bfseries,
  title=Key Idea,
  breakable,
  enhanced,
  #1
}

% Orange Box - Try It Yourself
\newtcolorbox{tryit}[1][]{
  colback=orange!10,
  colframe=orange!80!black,
  fonttitle=\bfseries,
  title=Try It Yourself,
  breakable,
  enhanced,
  #1
}

% Gray Box - Chapter Objectives
\newtcolorbox{chapterobjectives}[1][]{
  colback=gray!10,
  colframe=gray!70!black,
  fonttitle=\bfseries\Large,
  title=Chapter Objectives,
  breakable,
  enhanced,
  #1
}

% Level Boxes
\newtcolorbox{level1}[1][]{
  colback=level1green!10,
  colframe=level1green!80!black,
  fonttitle=\bfseries,
  title={[L1] Level 1: Intuitive Understanding},
  breakable,
  enhanced,
  before skip=10pt plus 2pt,
  after skip=10pt plus 2pt,
  #1
}

\newtcolorbox{level2}[1][]{
  colback=level2yellow!10,
  colframe=level2yellow!80!black,
  fonttitle=\bfseries,
  title={[L2] Level 2: Technical Details},
  breakable,
  enhanced,
  before skip=10pt plus 2pt,
  after skip=10pt plus 2pt,
  #1
}

\newtcolorbox{level3}[1][]{
  colback=level3red!10,
  colframe=level3red!80!black,
  fonttitle=\bfseries,
  title={[L3] Level 3: Research \& Verification},
  breakable,
  enhanced,
  before skip=10pt plus 2pt,
  after skip=10pt plus 2pt,
  #1
}

% Advanced box (alias for level3 for backward compatibility)
\newtcolorbox{advanced}[1][]{
  colback=level3red!10,
  colframe=level3red!80!black,
  fonttitle=\bfseries,
  title={[L3] Advanced Topic},
  breakable,
  enhanced,
  #1
}

% Observation box
\newtcolorbox{observation}[1][]{
  colback=cyan!5,
  colframe=cyan!60!black,
  fonttitle=\bfseries,
  title=Observation,
  breakable,
  enhanced,
  #1
}

% Conjecture box
\newtcolorbox{conjecture}[1][]{
  colback=thmgreen!5,
  colframe=thmgreen,
  fonttitle=\bfseries,
  title=Conjecture,
  breakable,
  enhanced,
  #1
}

% Problem box
\newtcolorbox{problem}[1][]{
  colback=exampleyellow!10,
  colframe=exampleyellow!80!black,
  fonttitle=\bfseries,
  title=Problem,
  breakable,
  enhanced,
  #1
}

% Axiom box
\newtcolorbox{axiom}[1][]{
  colback=defblue!5,
  colframe=defblue,
  fonttitle=\bfseries,
  title=Axiom,
  breakable,
  enhanced,
  #1
}

% Green box (generic green colored box)
\newtcolorbox{greenbox}[1][]{
  colback=thmgreen!5,
  colframe=thmgreen,
  fonttitle=\bfseries,
  title=Note,
  breakable,
  enhanced,
  #1
}

% Task box (for exercises/tasks)
\newtcolorbox{task}[1][]{
  colback=orange!10,
  colframe=orange!80!black,
  fonttitle=\bfseries,
  title=Task,
  breakable,
  enhanced,
  #1
}

% Intuition box (for intuitive explanations)
\newtcolorbox{intuition}[1][]{
  colback=blue!10,
  colframe=blue!70!black,
  fonttitle=\bfseries,
  title=Intuition,
  breakable,
  enhanced,
  #1
}

% Speculation box (for speculative ideas)
\newtcolorbox{speculation}[1][]{
  colback=purple!10,
  colframe=purple!70!black,
  fonttitle=\bfseries,
  title=Speculation,
  breakable,
  enhanced,
  #1
}

% ============================================
% CUSTOM COMMANDS
% ============================================

% Margin icons
\newcommand{\iconlevel}[1]{\marginpar{\Large #1}}
\newcommand{\iconintuitive}{\iconlevel{🟢}}
\newcommand{\icontechnical}{\iconlevel{🟡}}
\newcommand{\iconresearch}{\iconlevel{🔴}}
\newcommand{\iconcode}{\iconlevel{💻}}
\newcommand{\icongraph}{\iconlevel{📊}}
\newcommand{\iconimportant}{\iconlevel{🎯}}
\newcommand{\iconwarning}{\iconlevel{⚠️}}

% Mathematical operators and symbols
\newcommand{\R}{\mathbb{R}}
\newcommand{\C}{\mathbb{C}}
\newcommand{\N}{\mathbb{N}}
\newcommand{\Z}{\mathbb{Z}}
\newcommand{\Q}{\mathbb{Q}}

% Complex analysis operators
\DeclareMathOperator{\Log}{Log}  % Principal branch of logarithm
\DeclareMathOperator{\Arg}{Arg}  % Principal argument
\DeclareMathOperator{\Li}{Li}    % Polylogarithm
\DeclareMathOperator{\Real}{Re}  % Real part (alternative to \Re)
\DeclareMathOperator{\im}{Im}    % Imaginary part (alternative to \Im)

% Algebraic geometry and topology operators
\DeclareMathOperator{\Hdg}{Hdg}      % Hodge classes
\DeclareMathOperator{\Alg}{Alg}      % Algebraic classes
\DeclareMathOperator{\CH}{CH}        % Chow group
\DeclareMathOperator{\cl}{cl}        % Cycle class map
\DeclareMathOperator{\ch}{ch}        % Chern character
\DeclareMathOperator{\Ch}{Ch}        % Total Chern character
\DeclareMathOperator{\Hol}{Hol}      % Holomorphic
\DeclareMathOperator{\Spec}{Spec}    % Spectrum
\DeclareMathOperator{\Reg}{Reg}      % Regular part

% Algebra and group theory
\DeclareMathOperator{\SU}{SU}        % Special unitary group
\DeclareMathOperator{\ord}{ord}      % Order

% Computer science complexity classes
\DeclareMathOperator{\TIME}{TIME}    % Time complexity
\DeclareMathOperator{\NTIME}{NTIME}  % Nondeterministic time

% Number theory
\DeclareMathOperator{\Sha}{Sha}      % Shafarevich-Tate group

% General operators
\DeclareMathOperator{\encode}{encode}  % Encoding function
\newcommand{\F}{\mathbb{F}}            % Finite field

% Digital sum function
\newcommand{\dthree}[1]{D_3(#1)}

% Fractal Resonance Function
\newcommand{\Rf}[2]{R_f(#1,#2)}

% Consciousness field
\newcommand{\Cfield}{C^{\mu\nu}}

% Resonance coefficient
\newcommand{\rescoeff}[1]{\xi(#1)}

% Omega-space (already defined in LaTeX)
% \Omega is a built-in command

% ============================================
% THEOREM STYLES
% ============================================

\theoremstyle{plain}
\newtheorem{thm}{Theorem}[chapter]
\newtheorem{lemma}[thm]{Lemma}
\newtheorem{proposition}[thm]{Proposition}
\newtheorem{corollary}[thm]{Corollary}

\theoremstyle{definition}
\newtheorem{defn}[thm]{Definition}
\newtheorem{exmp}[thm]{Example}
\newtheorem{exercise}{Exercise}[chapter]

\theoremstyle{remark}
\newtheorem{remark}[thm]{Remark}
\newtheorem{note}[thm]{Note}

% ============================================
% CODE LISTING SETTINGS
% ============================================

\lstset{
  basicstyle=\ttfamily\small,
  keywordstyle=\color{blue}\bfseries,
  commentstyle=\color{gray}\itshape,
  stringstyle=\color{red},
  showstringspaces=false,
  breaklines=true,
  frame=single,
  numbers=left,
  numberstyle=\tiny\color{gray},
  backgroundcolor=\color{gray!10}
}

% Python settings
\lstdefinestyle{python}{
  language=Python,
  morekeywords={numpy,scipy,mpmath}
}

% ============================================
% HYPERREF SETTINGS
% ============================================

\hypersetup{
  colorlinks=true,
  linkcolor=blue,
  citecolor=thmgreen,
  urlcolor=defblue,
  pdftitle={Principia Fractalis: A New Mathematical Framework for Physics, Computation, and Consciousness},
  pdfauthor={Pablo Cohen},
  pdfsubject={Mathematics, Physics, Consciousness},
  pdfkeywords={Fractal Resonance, Riemann Hypothesis, Consciousness, Quantum Field Theory}
}

% ============================================
% HEADER AND FOOTER
% ============================================

\pagestyle{fancy}
\fancyhf{}
\fancyhead[LE]{\leftmark}
\fancyhead[RO]{\rightmark}
\fancyfoot[C]{\thepage}
\renewcommand{\headrulewidth}{0.4pt}
\renewcommand{\footrulewidth}{0pt}

% ============================================
% TITLE INFORMATION
% ============================================

\title{Fractal Resonance Mathematics\\The Definitive Textbook}
\author{Pablo Cohen\\ORCID: 0009-0002-0734-5565}
\date{\today}

% ============================================
% MISCELLANEOUS
% ============================================

% Line spacing
\onehalfspacing

% Chapter opening customization - Enhanced for v2.0
\usepackage{titlesec}
\titleformat{\chapter}[display]
  {\normalfont\Huge\bfseries\color{pfblue}}
  {\filleft\Huge\color{pfgold}\chaptertitlename\ \thechapter}
  {4ex}
  {\titlerule\vspace{2ex}\filleft}
  [\vspace{2ex}\titlerule]

% Part formatting - Enhanced for v2.0
\titleformat{\part}[display]
  {\normalfont\Huge\bfseries\centering\color{pfblack}}
  {\thepart}
  {20pt}
  {\Huge\color{pfgold}}

% Better page breaking for boxes
\usepackage{needspace}
\newcommand{\boxneedspace}{\needspace{10\baselineskip}}

% Better table spacing
\renewcommand{\arraystretch}{1.3}

% End of preamble
